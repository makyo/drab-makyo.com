In my time working in software, I have found that nothing matters quite so much as the end user. This goes far beyond just producing software that runs and does what it says on the tin. It means taking into account just how easy it is for the end user to write their software, just how much information is available to them in order to ensure everything works smoothly, and just how responsive the developer (whether individual or corporation) is to their needs and wants.

The problem, then is one of education and engagement. Without the ability to provide the information the users need in a clear, concise, and friendly fashion, they are going to have little to work off of. Without engaging with them, they are going to feel left in the dark, unable to figure out just what it is that they can expect from the software --- after all, if there is only radio silence on the other end of the tool, then it is easy to feel abandoned.

Canonical, as I am certainly well aware as a former employee, is a company that prides itself on being responsive and informative about the software it makes, supports, and uses. The effort of AskUbuntu alone is enough to show this, but having worked with the Snap Store and with Juju charms in the past, it is plain to see for for anyone.

I have been giving lectures and presentations for going on half a decade now and writing informative posts and essays for twice that, and being able to teach and inform, to take questions and answer them fluently and ensure that the asker gets what they need, is something that I truly enjoy. Given the alignment of goals and passions, it is only my hope that I can help bring that to Canonical as well.

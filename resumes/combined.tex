\documentclass[letterpaper]{memoir}

\usepackage[top=1in, left=1in, bottom=1in, right=1in]{geometry}
\pagestyle{empty}
\renewcommand{\baselinestretch}{1.25}

\usepackage{PlayfairDisplay}
\usepackage[T1]{fontenc}
\renewcommand*\oldstylenums[1]{{\playfairOsF #1}}

\usepackage[usenames, dvipsnames]{color}
\definecolor{titlegreylight}{gray}{0.5}
\definecolor{titlegreydark}{gray}{0.4}

\usepackage{wrapfig}

\begin{document}


\begin{center}
  {\huge \textit{\color{titlegreydark} Madison Jesse Scott-Clary}}\\
  {\Huge{\textbf{R\'esum\'e}}}\\
  {\color{titlegreydark} makyo@drab-makyo.com $\bullet$ 3Sultan, WA $\bullet$ 303-818-5943}\\
  {\small \color{titlegreylight} \textit{Full r\'esum\'e and C.V. available at makyo.io/resume}}
\end{center}


\textit{\large Madison Scott-Clary} is an author, technical writer, and editor living in Loveland, CO. She works heavily with both fiction and non-fiction in an authorial and editorial role. She strives to further her knowledge within both areas, as well as to expand into other fields both within and outside of writing and editing. As a supporter and proponent of minority gender identities and sexual orientations, she pushes for positive representation and healthy role models in written works.

\subsection{Skills}

\begin{wrapfigure}{r}{0.35\linewidth}
  \textit{\color{titlegreydark} \large Madison cares deeply about making it as easy as possible for writers to interface with editors and publishers, working to find ways to let the authors and editors to focus on their respective jobs. As a trans woman, she is focused on gender representation in both the writing industry and within the authorial community, knowing that those exploring their identity need positive examples and role-models to look up to.}
  \vspace{-1in}
\end{wrapfigure}

\textbf{Line editing}

\begin{itemize}
  \item Firm grasp on English as a written language, and American and British English as dialects
  \item Experience with style manuals (Chicago and AP)
  \item Understanding of stylistic concerns and genre within fiction
  \item Experience with anthologies and collections of both fiction and non-fiction
  \item Experience sharing edits and managing changes in Microsoft Word, LibreOffice, \LaTeX, plain text, and PDF annotation, as well as paper editing
\end{itemize}

\hspace{-1.5em}\textbf{Content editing}

\begin{itemize}
  \item Experienced in writing and promoting calls for submissions in a clear format that invite authors to submit
  \item Comfortable evaluating submitted work on its evocativeness, marketability, fixability, and fit
  \item Insistent on clear communication with authors as an editor, expressing needs and suggestions to work towards a finished product
\end{itemize}

\hspace{-1.5em}\textbf{Copyediting and Layout}

\begin{itemize}
  \item Well-versed in \LaTeX\ and some experience with Adobe InDesign for layout
  \item Experienced with publishing online for both rich-media and static presentation in WordPress, Ghost, and Jekyll, for both fiction and non-fiction
  \item Experience publishing for print and e-reader formats as well as the web
\end{itemize}

\newpage

\textit{\large Madison Scott-Clary} is also a software engineer, working heavily with web development, both on the front- and back-end, as well as with DevOps and cloud-based solutions. She strives to further her knowledge within both areas, as well as to expand into other fields both within and outside of software engineering and computers. As a supporter and proponent of Free and Open-Source Software, she is committed to providing the best tools and products that she can using FOSS and under OSI-approved licenses where appropriate.

\begin{wrapfigure}{l}{0.4\linewidth}
  \textit{\color{titlegreydark} \large As a process maven, Madison is experienced and adamant about CI, test coverage and reporting, and code documentation. She has also worked with technical writing, blogging, and promotional writing for software products She has experience working in a management, mediator, and technical lead capacity, and is keenly interested in a diverse workforce, healthy work-life balance, and constructive social interactions within the workplace. As a trans woman, she is very focused on gender representation in tech.}
  \vspace{-0.7in}
\end{wrapfigure}

\textbf{Web development}

\begin{itemize}
  \item \textit{Back-end technologies:} Python (using Flask and Django), Go (using Martini and Gorilla)
  \item \textit{Front-end technologies:} standards compliant HTML5, CSS (standard, Sass, and Less), and JavaScript (vanilla, CoffeeScript, and Elm)
  \item Focus on data visualization with D3, nvd3, React, and Crossfilter
  \item Responsive application design through frameworks (Vanilla, Bootstrap, etc) or bespoke styling.
\end{itemize}

\hspace{-1.5em}\textbf{Cloud and DevOps}

\begin{itemize}
  \item Experince with programatic deployment using juju with cloud providers: Amazon AWS, Google Compute Engine, and Microsoft Azure; local providers: LXD, MaaS, and manual providers based on discrete servers; and ancillary services: Digital Ocean, Linode and Rackspace
  \item providing highly available, load balanced, and monitored applications and application stacks
  \item deployment strategies such as Juju, but also Chef, Puppet, and Ansible
\end{itemize}

\hspace{-1.5em}\textbf{Systems and Languages}

\begin{itemize}
  \item \textit{Fluent in:} Python, Javascript, Make, Flask, Django, D3
  \item \textit{Comfortable in:} Go, Ruby, Coffeescript, Bash, Martini, nvd3, YUI
  \item \textit{New to:} Elm, React, Rails
  \item Fluent in Linux operation on both desktop and server, as well as macOS and Windows
\end{itemize}

\newpage

\subsection{Editorial work experience}

\begin{description}
\item[Hybrid Ink, LLC]
Editor-in-chief --- 2018--\emph{present}

\begin{itemize}
\tightlist
\item
  Shepherded publications --- both fiction and non --- from the query process through final publication and sales.
\item
  Worked with authors through the editorial and promotion process.
\item
  Created and maintained several advertising and sales channels
\end{itemize}
\item[Thurston Howl Publications]
Editor --- 2017--2018

\begin{itemize}
\tightlist
\item
  Worked on fiction and non-fiction anthologies of shorter works
\item
  Fielded queries for longer works, including judging the query and
  providing an initial read-through of the manuscript
\item
  Contributed both fiction and non-fiction writing to anthologies
\item
  Copyedited and formatted books, including overall layout in (Xe)LaTeX, LibreOffice, and Word
\end{itemize}
\item[Furry Writers' Guild] --- President --- 2017

\begin{itemize}
  \item Helped to guide and direct the Guild, including setting attainable goals, working with membership management, and managing services provided to members and non-members alike
  \item Helped promote the guild and its activities to a wider audience, including as Guest-of-Honor at Furry Migration 2017
  \item Worked with guild members in an editorial, publishing, beta-read, and fellow writer capacity
\end{itemize}
\item[{[}adjective{]}{[}species{]}, Ltd.]
Editor-in-chief --- 2011--\emph{present}

\begin{itemize}
\tightlist
\item
  Created, designed, and ran the websites adjectivespecies.com,
  lovesexfur.com, and furrypoll.com
\item
  Fielded queries and submissions of primarily short non-fiction works
  to be published on the sites
\item
  Contributed non-fiction writing to the sites
\item
  Produced a set of informational guides to be handed out and used as
  the basis of presentations
\item
  Spoke as a panelist at several conventions on the topic of data, safer
  sex, relationships, and gender
\item
  Created and helped others create small surveys to collect data to be
  used as the basis for articles and presentations
\item
  Took over administration of The Furry Poll (previously The Furry
  Survey) and ran a longitudinal survey of the furry subculture over
  several years
\end{itemize}
\end{description}

\newpage

\subsection{Tech work experience}

\begin{description}
\item[Canonical, Ltd.]
Software engineer --- 2012--\emph{present}

\begin{itemize}
\tightlist
\item
  Shepherded the Juju GUI from inception to production, working to implement changes from the core API as it moved from Python to Go
\item
  Worked to implement tooling around bundles of charms on various levels:

  \begin{itemize}
  \tightlist
  \item
    Python-based Juju Quickstart, which created a model and deployed charms to it in Juju 1
  \item
    Python-based juju-bundlelib library, which broke a bundle down into the composite steps needed to deploy it (bundle changes), information that can be consumed by the Juju GUI
  \item
    Go-based jujusvg library, which generated an SVG image of a bundle as it would appear on the Juju GUI's canvas
  \item
    Go-based bundleservice and corresponding charm, which provided an API endpoint for generating both the bundle changes and bundle SVG
  \end{itemize}
  \item
    Implemented client libraries in both JavaScript and Python for interacting with varied microservices.
  \item
    Helped with continued support of projects and their corresponding charms, such as the Extended Support Mechanism allowing for support for older versions of Ubuntu, and the Livepatch service allowing for serving kernel patches to be applied without reboot.
\end{itemize}
\item[bConnected Software/Optum Health/United Health Group]
Software engineer --- 2011--2012

\begin{itemize}
\tightlist
\item
  Worked to implement a supplemental insurance sales portal from initial meetings to release
\item
  Developed web applications in Grails and JavaScript, backed by an API provided by an in-house tool
\item
  Designed and implemented an XML-based rules engine for calculating both plan rates and eligibility within the web applications, extending the Java-based in-house tool
\item
  Built a basic editor for the rules engine and a related forms engine
\end{itemize}
\item[Colorado State University Libraries]
Library Technical Services/Research \& Development --- 2007--2011

\begin{itemize}
\tightlist
\item
  Provided hardware and software technical support to library staff, comprising 400-500 desktop machines
\item
  Worked with specialized equipment such as flat book scanners, archival scanners, and plate-glass negative scanners
\item
  Managed the fleet of \textasciitilde{}300 desktops and \textasciitilde{}200 laptops available for public use
\item
  Developed in-house software for managing the Atmospheric Sciences Reading Room, a branch library, allowing basic lending of materials
\item
  Investigated custom software for mapping resources in the library as well as locations around the campus, providing shortest-path routing from current location
\end{itemize}
\end{description}

\subsection{Education}\label{education}

\begin{description}
\tightlist
\item[University] \hfill
Colorado State University (2004-2011) studying music composition and
computer science.
\end{description}

Additional employment experience, publications, and volunteer projects
Madison has worked with are available on her C.V.

\end{document}

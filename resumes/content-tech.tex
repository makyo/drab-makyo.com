\subsection{Skills}

\begin{wrapfigure}{r}{0.4\linewidth}
  \textit{\color{titlegreydark} \large As a process maven, Madison is experienced and adamant about CI, test coverage and reporting, and code documentation. She has also worked with technical writing, blogging, and promotional writing for software products She has experience working in a management, mediator, and technical lead capacity, and is keenly interested in a diverse workforce, healthy work-life balance, and constructive social interactions within the workplace. As a trans woman, she is very focused on gender representation in tech.}
  \vspace{-1in}
\end{wrapfigure}

\textbf{Web development}

\begin{itemize}
  \item \textit{Back-end technologies:} Python (using Flask and Django), Go (using Martini and Gorilla)
  \item \textit{Front-end technologies:} standards compliant HTML5, CSS (standard, Sass, and Less), and JavaScript (vanilla, CoffeeScript, and Elm)
  \item Focus on data visualization with D3, nvd3, React, and Crossfilter
  \item Responsive application design through frameworks (Vanilla, Bootstrap, etc) or bespoke styling.
\end{itemize}

\hspace{-1.5em}\textbf{Cloud and DevOps}

\begin{itemize}
  \item Experince with programatic deployment using juju with cloud providers: Amazon AWS, Google Compute Engine, and Microsoft Azure; local providers: LXD, MaaS, and manual providers based on discrete servers; and ancillary services: Digital Ocean, Linode and Rackspace
  \item providing highly available, load balanced, and monitored applications and application stacks
  \item deployment strategies such as Juju, but also Chef, Puppet, and Ansible
\end{itemize}

\newpage

\subsection{Work experience}

\begin{description}
\item[Canonical, Ltd.]
Software engineer --- 2012-\emph{present}

\begin{itemize}
\tightlist
\item
  Shepherded the Juju GUI from inception to production, working to implement changes from the core API as it moved from Python to Go
\item
  Worked to implement annotation information within Juju core to allow persistent GUI state to be stored within Juju models
\item
  Worked to implement tooling around bundles of charms on various levels:

  \begin{itemize}
  \tightlist
  \item
    Python-based Juju Quickstart, which created a model and deployed charms to it in Juju 1
  \item
    Python-based juju-bundlelib library, which broke a bundle down into the composite steps needed to deploy it (bundle changes), information that can be consumed by the Juju GUI
  \item
    Go-based jujusvg library, which generated an SVG image of a bundle as it would appear on the Juju GUI's canvas
  \item
    Go-based bundleservice and corresponding charm, which provided an API endpoint for generating both the bundle changes and bundle SVG
  \end{itemize}
\end{itemize}
\item[bConnected Software/Optum Health/United Health Group]
Software engineer --- 2011-2012

\begin{itemize}
\tightlist
\item
  Worked to implement a supplemental insurance sales portal from initial meetings to release
\item
  Developed web applications in Grails and JavaScript, backed by an API provided by an in-house tool
\item
  Designed and implemented an XML-based rules engine for calculating both plan rates and eligibility within the web applications
\item
  Extended the Java-based in-house tool to work with the rules engine
\item
  Built a basic editor for the rules engine and a related forms engine
\end{itemize}
\item[Colorado State University Libraries]
Library Technical Services/Research \& Development --- 2007-2011

\begin{itemize}
\tightlist
\item
  Provided hardware and software technical support to library staff, comprising 400-500 desktop machines
\item
  Worked with specialized equipment such as flat book scanners, archival scanners, and plate-glass negative scanners
\item
  Managed the fleet of \textasciitilde{}300 desktops and \textasciitilde{}200 laptops available for public use
\item
  Developed in-house software for managing the Atmospheric Sciences Reading Room, a branch library, allowing basic lending of materials
\item
  Investigated custom software for mapping resources in the library as well as locations around the campus, providing shortest-path routing from current location
\end{itemize}
\end{description}
